% ---------- Embedded Bibliography ----------
\begin{filecontents*}[overwrite]{\jobname.bib}
@article{aslam2023,
  author  = {Aslam, A. and Kwo, P. Y.},
  title   = {Epidemiology and disease burden of alcohol associated liver disease},
  journal = {Journal of Clinical and Experimental Hepatology},
  year    = {2023},
  volume  = {13},
  pages   = {88--102}
}

@article{guirguis2015,
  author  = {Guirguis, J. and Chhatwal, J. and Dasarathy, J. and Rivas, J. and McMichael, D. and Nagy, L. E. and others},
  title   = {Clinical impact of alcohol-related cirrhosis in the next decade: Estimates based on current epidemiological trends in the United States},
  journal = {Alcoholism: Clinical and Experimental Research},
  year    = {2015},
  volume  = {39},
  pages   = {2085--2094}
}

@article{huang2023,
  author  = {Huang, D. Q. and Mathurin, P. and Cortez-Pinto, H. and Loomba, R.},
  title   = {Global epidemiology of alcohol-associated cirrhosis and HCC: Trends, projections and risk factors},
  journal = {Nature Reviews Gastroenterology \& Hepatology},
  year    = {2023},
  volume  = {20},
  pages   = {37--49}
}

@article{niu2023,
  author  = {Niu, X. and Zhu, L. and Xu, Y. and Zhang, M. and Hao, Y. and Ma, L. and others},
  title   = {Global prevalence, incidence, and outcomes of alcohol related liver diseases: A systematic review and meta-analysis},
  journal = {BMC Public Health},
  year    = {2023},
  volume  = {23},
  pages   = {859}
}

@article{sengupta2024,
  author  = {Sengupta, S. and Anand, A. and Lopez, R. and Weleff, J. and Wang, P. R. and Bellar, A. and others},
  title   = {Emergency services utilization by patients with alcohol-associated hepatitis: An analysis of national trends},
  journal = {Alcohol: Clinical and Experimental Research},
  year    = {2024},
  volume  = {48},
  pages   = {98--109}
}

@article{hosseini2019,
  author  = {Hosseini, N. and Shor, J. and Szabo, G.},
  title   = {Alcoholic hepatitis: A review},
  journal = {Alcohol and Alcoholism},
  year    = {2019},
  volume  = {54},
  pages   = {408--416}
}

@article{osna2017,
  author  = {Osna, N. A. and Donohue, T. M. and Kharbanda, K. K.},
  title   = {Alcoholic liver disease: Pathogenesis and current management},
  journal = {Alcohol Research},
  year    = {2017},
  volume  = {38},
  pages   = {147--161}
}

@article{tu2023,
  author  = {Tu, W. and Gawrieh, S. and Dasarathy, S. and Mitchell, M. C. and Simonetto, D. A. and Patidar, K. R. and others},
  title   = {Design of a multicenter randomized clinical trial for treatment of Alcohol-Associated Hepatitis},
  journal = {Contemporary Clinical Trials Communications},
  year    = {2023},
  volume  = {32},
  pages   = {101074}
}

@article{dasarathy2020,
  author  = {Dasarathy, S. and Mitchell, M. C. and Barton, B. and McClain, C. J. and Szabo, G. and Nagy, L. E. and others},
  title   = {Design and rationale of a multicenter defeat alcoholic steatohepatitis trial (DASH): Randomized clinical trial to treat alcohol-associated hepatitis},
  journal = {Contemporary Clinical Trials},
  year    = {2020},
  volume  = {96},
  pages   = {106094}
}

@article{forrest2013,
  author  = {Forrest, E. and Mellor, J. and Stanton, L. and Bowers, M. and Ryder, P. and Austin, A. and others},
  title   = {Steroids or pentoxifylline for alcoholic hepatitis (STOPAH): Study protocol for a randomised controlled trial},
  journal = {Trials},
  year    = {2013},
  volume  = {14},
  pages   = {262}
}

@article{thursz2015,
  author  = {Thursz, M. R. and Richardson, P. and Allison, M. and Austin, A. and Bowers, M. and Day, C. P. and others},
  title   = {Prednisolone or pentoxifylline for alcoholic hepatitis},
  journal = {New England Journal of Medicine},
  year    = {2015},
  volume  = {372},
  pages   = {1619--1628}
}

@article{szabo2022,
  author  = {Szabo, G. and Mitchell, M. and McClain, C. J. and Dasarathy, S. and Barton, B. and McCullough, A. J. and others},
  title   = {IL-1 receptor antagonist plus pentoxifylline and zinc for severe alcohol-associated hepatitis},
  journal = {Hepatology},
  year    = {2022},
  volume  = {76},
  pages   = {1058--1068}
}

@article{gawrieh2024,
  author  = {Gawrieh, S. and Dasarathy, S. and Tu, W. and Kamath, P. S. and Chalasani, N. P. and McClain, C. J. and others},
  title   = {Randomized trial of anakinra plus zinc vs. prednisone for severe alcohol-associated hepatitis},
  journal = {Journal of Hepatology},
  year    = {2024},
  volume  = {80},
  pages   = {684--693}
}

@article{lee2024,
  author  = {Lee, B. P. and Witkiewitz, K. and Mellinger, J. and Anania, F. A. and Bataller, R. and Cotter, T. G. and others},
  title   = {Designing clinical trials to address alcohol use and alcohol-associated liver disease: An expert panel Consensus Statement},
  journal = {Nature Reviews Gastroenterology \& Hepatology},
  year    = {2024},
  volume  = {21},
  pages   = {626--645}
}

@article{liangpunsakul2011,
  author  = {Liangpunsakul, S.},
  title   = {Clinical characteristics and mortality of hospitalized alcoholic hepatitis patients in the United States},
  journal = {Journal of Clinical Gastroenterology},
  year    = {2011},
  volume  = {45},
  pages   = {714--719}
}

@article{kaur2022,
  author  = {Kaur, B. and Rosenblatt, R. and Sundaram, V.},
  title   = {Infections in alcoholic hepatitis},
  journal = {Journal of Clinical and Translational Hepatology},
  year    = {2022},
  volume  = {10},
  pages   = {718--725}
}

@article{diCola2023,
  author  = {Di Cola, S. and Gazda, J. and Lapenna, L. and Ceccarelli, G. and Merli, M.},
  title   = {Infection prevention and control programme and COVID-19 measures: Effects on hospital-acquired infections in patients with cirrhosis},
  journal = {JHEP Reports},
  year    = {2023},
  volume  = {5},
  pages   = {100703}
}

@article{lourens2017,
  author  = {Lourens, S. and Sunjaya, D. B. and Singal, A. and Liangpunsakul, S. and Puri, P. and Sanyal, A. and others},
  title   = {Acute alcoholic hepatitis: Natural history and predictors of mortality using a multicenter prospective study},
  journal = {Mayo Clinic Proceedings: Innovations, Quality \& Outcomes},
  year    = {2017},
  volume  = {1},
  pages   = {37--48}
}

@article{shang2025,
  author  = {Shang, Y. and Akbari, C. and Dodd, M. and Zhang, X. and Wang, T. and Jemielita, T. and others},
  title   = {Association between longitudinal biomarkers and major adverse liver outcomes in patients with non-cirrhotic metabolic dysfunction-associated steatotic liver disease},
  journal = {Hepatology},
  year    = {2025},
  volume  = {81},
  pages   = {1501--1511}
}

@article{dagPerf2024,
  author  = {Dasarathy, S. and Tu, W. and Bellar, A. and Welch, N. and Kettler, C. and Tang, Q. and others},
  title   = {Development and evaluation of objective trial performance metrics for multisite clinical studies: Experience from the AlcHep Network},
  journal = {Contemporary Clinical Trials},
  year    = {2024},
  volume  = {138},
  pages   = {107437}
}

@article{crabb2016,
  author  = {Crabb, D. W. and Bataller, R. and Chalasani, N. P. and Kamath, P. S. and Lucey, M. and Mathurin, P. and others},
  title   = {Standard definitions and common data elements for clinical trials in patients with alcoholic hepatitis: Recommendation from the NIAAA Alcoholic Hepatitis Consortia},
  journal = {Gastroenterology},
  year    = {2016},
  volume  = {150},
  pages   = {785--790}
}

@article{louvet2017,
  author  = {Louvet, A. and Labreuche, J. and Artru, F. and Bouthors, A. and Rolland, B. and Saffers, P. and others},
  title   = {Main drivers of outcome differ between short term and long term in severe alcoholic hepatitis: A prospective study},
  journal = {Hepatology},
  year    = {2017},
  volume  = {66},
  pages   = {1464--1473}
}

@article{connor2023,
  author  = {Connor, L. and Dean, J. and McNett, M. and Tydings, D. M. and Shrout, A. and Gorsuch, P. F. and others},
  title   = {Evidence-based practice improves patient outcomes and healthcare system return on investment: Findings from a scoping review},
  journal = {Worldviews on Evidence-Based Nursing},
  year    = {2023},
  volume  = {20},
  pages   = {6--15}
}

@article{kamath2001,
  author  = {Kamath, P. S. and Wiesner, R. H. and Malinchoc, M. and Kremers, W. and Therneau, T. M. and Kosberg, C. L. and others},
  title   = {A model to predict survival in patients with end-stage liver disease},
  journal = {Hepatology},
  year    = {2001},
  volume  = {33},
  pages   = {464--470}
}

@article{singal2013,
  author  = {Singal, A. K. and Kamath, P. S.},
  title   = {Model for end-stage liver disease},
  journal = {Journal of Clinical and Experimental Hepatology},
  year    = {2013},
  volume  = {3},
  pages   = {50--60}
}

@article{jophlin2024,
  author  = {Jophlin, L. L. and Singal, A. K. and Bataller, R. and Wong, R. J. and Sauer, B. G. and Terrault, N. A. and others},
  title   = {ACG clinical guideline: Alcohol-associated liver disease},
  journal = {American Journal of Gastroenterology},
  year    = {2024},
  volume  = {119},
  pages   = {30--54}
}

@article{hsu2023,
  author  = {Hsu, C. C. and Dodge, J. L. and Weinberg, E. and Im, G. and Ko, J. and Davis, W. and others},
  title   = {Multicentered study of patient outcomes after declined for early liver transplantation in severe alcohol-associated hepatitis},
  journal = {Hepatology},
  year    = {2023},
  volume  = {77},
  pages   = {1253--1262}
}

@article{musto2022,
  author  = {Musto, J. and Stanfield, D. and Ley, D. and Lucey, M. R. and Eickhoff, J. and Rice, J. P.},
  title   = {Recovery and outcomes of patients denied early liver transplantation for severe alcohol-associated hepatitis},
  journal = {Hepatology},
  year    = {2022},
  volume  = {75},
  pages   = {104--114}
}

@article{pitt2021,
  author  = {Pitt, B. and Filippatos, G. and Agarwal, R. and Anker, S. D. and Bakris, G. L. and Rossing, P. and others},
  title   = {Cardiovascular events with finerenone in kidney disease and type 2 diabetes},
  journal = {New England Journal of Medicine},
  year    = {2021},
  volume  = {385},
  pages   = {2252--2263}
}

@article{bakris2020,
  author  = {Bakris, G. L. and Agarwal, R. and Anker, S. D. and Pitt, B. and Ruilope, L. M. and Rossing, P. and others},
  title   = {Effect of finerenone on chronic kidney disease outcomes in type 2 diabetes},
  journal = {New England Journal of Medicine},
  year    = {2020},
  volume  = {383},
  pages   = {2219--2229}
}

@article{sprintrl2015,
  author  = {{SPRINT Research Group}},
  title   = {A randomized trial of intensive versus standard blood-pressure control},
  journal = {New England Journal of Medicine},
  year    = {2015},
  volume  = {373},
  pages   = {2103--2116}
}

@article{dunn2005,
  author  = {Dunn, W. and Jamil, L. H. and Brown, L. S. and Wiesner, R. H. and Kim, W. R. and Menon, K. N. and others},
  title   = {MELD accurately predicts mortality in patients with alcoholic hepatitis},
  journal = {Hepatology},
  year    = {2005},
  volume  = {41},
  pages   = {353--358}
}

@article{forrest2005,
  author  = {Forrest, E. and Evans, C. and Stewart, S. and Phillips, M. and Oo, Y. H. and McAvoy, N. and others},
  title   = {Analysis of factors predictive of mortality in alcoholic hepatitis and derivation and validation of the Glasgow alcoholic hepatitis score},
  journal = {Gut},
  year    = {2005},
  volume  = {54},
  pages   = {1174--1179}
}

@article{pang2015,
  author  = {Pang, J. X. and Ross, E. and Borman, M. A. and Zimmer, S. and Kaplan, G. G. and Heitman, S. J. and others},
  title   = {Risk factors for mortality in patients with alcoholic hepatitis and assessment of prognostic models: A population-based study},
  journal = {Canadian Journal of Gastroenterology and Hepatology},
  year    = {2015},
  volume  = {29},
  pages   = {131--138}
}

@article{kasztelan2013,
  author  = {Kasztelan{-}Szczerbi{\'n}ska, B. and S{\l}omka, M. and Celinski, K. and Szczerbinski, M.},
  title   = {Alkaline phosphatase: The next independent predictor of the poor 90-day outcome in alcoholic hepatitis},
  journal = {BioMed Research International},
  year    = {2013},
  volume  = {2013},
  pages   = {614081}
}

@article{bennett2019,
  author  = {Bennett, K. and Enki, D. G. and Thursz, M. and Cramp, M. E. and Dhanda, A. D.},
  title   = {Systematic review with meta-analysis: High mortality in patients with non-severe alcoholic hepatitis},
  journal = {Alimentary Pharmacology \& Therapeutics},
  year    = {2019},
  volume  = {50},
  pages   = {249--257}
}

@article{musto2021,
  author  = {Musto, J. A. and Eickhoff, J. and Ventura{-}Cots, M. and Abraldes, J. G. and Bosques{-}Padilla, F. and Verna, E. C. and others},
  title   = {The level of alcohol consumption in the prior year does not impact clinical outcomes in patients with alcohol-associated hepatitis},
  journal = {Liver Transplantation},
  year    = {2021},
  volume  = {27},
  pages   = {1382--1391}
}

@article{hofer2023,
  author  = {Hofer, B. S. and Simbrunner, B. and Hartl, L. and Jachs, M. and Balcar, L. and Paternostro, R. and others},
  title   = {Hepatic recompensation according to Baveno VII criteria is linked to a significant survival benefit in decompensated alcohol-related cirrhosis},
  journal = {Liver International},
  year    = {2023},
  volume  = {43},
  pages   = {2220--2231}
}

@article{louvet2023,
  author  = {Louvet, A. and Labreuche, J. and Dao, T. and Thevenot, T. and Oberti, F. and Bureau, C. and others},
  title   = {Effect of prophylactic antibiotics on mortality in severe alcohol-related hepatitis: A randomized clinical trial},
  journal = {JAMA},
  year    = {2023},
  volume  = {329},
  pages   = {1558--1566}
}
\end{filecontents*}

\documentclass[12pt]{article}

% ---------- Packages ----------
\usepackage[margin=1in]{geometry}
\usepackage[T1]{fontenc}
\usepackage[utf8]{inputenc}
\usepackage[strict]{csquotes}
\usepackage{mathptmx}  % Times Roman font (NEJM requirement)
\usepackage{setspace}
\usepackage{authblk}
\usepackage{amsmath, amssymb}
\usepackage{siunitx}
\usepackage{threeparttable}
\usepackage{longtable,threeparttablex,booktabs}
\usepackage{graphicx}
\usepackage{caption}
\usepackage{subcaption}
\usepackage[hidelinks]{hyperref}
\usepackage{enumitem}
\usepackage[backend=biber,style=nejm,sorting=none,maxnames=6,minnames=3,terseinits=true,isbn=false]{biblatex}
\addbibresource{\jobname.bib}
\captionsetup{font=small, labelfont=bf}
\usepackage{tabularx}
\usepackage{lineno}

% ---------- Formatting tweaks ----------
\setlength{\parskip}{0.6em}
\setlength{\parindent}{0pt}
\doublespacing
\linenumbers

% ---------- Title & Authors ----------
\title{Natural history and development of a novel composite endpoint in patients with alcohol-associated hepatitis: Data from a prospective multicenter study}

\author[1]{\textbf{Srinivasan Dasarathy}}
\author[2]{\textbf{Wanzhu Tu}}
\author[1]{\textbf{Nicole Welch}}
\author[2]{\textbf{Samer Gawrieh}}
\author[2]{\textbf{Yunpeng Fu}}
\author[2]{\textbf{Qing Tang}}
\author[2]{\textbf{Carla Kettler}}
\author[3]{\textbf{Arun J.\ Sanyal}}
\author[4]{\textbf{Gyongyi Szabo}}
\author[5]{\textbf{Vijay H.\ Shah}}
\author[6]{\textbf{Ramon Bataller}}
\author[1]{\textbf{Laura E.\ Nagy}}
\author[7,8]{\textbf{Craig J.\ McClain}}
\author[2]{\textbf{Naga Chalasani}}
\author[9]{\textbf{Thomas Kerr}}
\author[9]{\textbf{Mack C.\ Mitchell}}

\affil[1]{Department of Gastroenterology and Hepatology, and Inflammation and Immunity, Cleveland Clinic, Cleveland, Ohio, USA}
\affil[2]{Department of Biostatistics and Health Data Science, Department of Medicine, Indiana University, Indianapolis, Indiana, USA}
\affil[3]{Department of Gastroenterology, Virginia Commonwealth University, Richmond, Virginia, USA}
\affil[4]{Department of Gastroenterology, Beth Israel Deaconess Medical Center, Boston, Massachusetts, USA}
\affil[5]{Department of Gastroenterology and Hepatology, Mayo Clinic, Rochester, Minnesota, USA}
\affil[6]{Clinic Barcelona, Barcelona, Spain}
\affil[7]{Department of Medicine, University of Louisville, Louisville, Kentucky, USA}
\affil[8]{Robley Rex Louisville VA Medical Center, Louisville, Kentucky, USA}
\affil[9]{Department of Gastroenterology and Hepatology, UT Southwestern, Dallas, Texas, USA}

\date{}

% ---------- Document ----------
\begin{document}
\maketitle

% Correspondence
\section*{Correspondence}
Srinivasan Dasarathy, Department of Gastroenterology and Hepatology, and Inflammation and Immunity, 9500 Euclid Avenue, NE4 208, Lerner Research Institute, Cleveland Clinic, Cleveland, OH 44195, USA. Email: \href{mailto:dasaras@ccf.org}{dasaras@ccf.org}

% Abstract
\begin{abstract}
\noindent\textbf{Background \& Aims:} The clinical course and outcomes of alcohol-associated hepatitis (AH) remain poorly understood. Major adverse liver outcomes do not capture the added risk of return to drinking. We examined the natural history of AH and developed a composite endpoint using a contemporary observational cohort of AH.

\noindent\textbf{Approach \& Results:} A cohort of 1{,}127 participants---712 AH patients, 256 heavy drinking controls without clinically evident liver disease, and 159 healthy controls---were prospectively followed for 6 months at 8 United States centers as part of the Alcoholic Hepatitis Network (AlcHepNet) consortium. Outcomes included mortality and a composite endpoint (AlcHepNet composite index) that included death, liver transplantation, hepatic decompensation (new onset/worsening ascites, HE, variceal bleeding), liver-related hospital admission, MELD increase $\ge 5$, and return to drinking. Of 712 AH patients (age $45 \pm 10.7$ y; 59.1\% male), 558 (79.0\%) had severe and 148 (21.0\%) had moderate AH; 232 (32.5\%) died, and 86 (12.1\%) underwent liver transplantation. Mortality rates in moderate AH and severe AH were 0.7\% versus 17.2\% (30 d), 3.4\% versus 26.5\% (90 d), and 8.8\% versus 30.5\% (180 d), respectively (all $p<0.001$). Composite liver/alcohol-use events were noted in 459 (64.5\%) AH patients. Higher MELD score, lower mean arterial pressure, and baseline leukocytosis were associated with higher 90-day mortality in AH (all $p<0.05$). College education and higher ALP were associated with lower mortality. Heavy drinking controls had low mortality ($n=3$; 1.2\%).

\noindent\textbf{Conclusions:} This large observational study showed a high incidence of composite liver and alcohol-use events within 6 months, reiterating the need for early interventions.

\noindent\textbf{Keywords:} alcohol-associated hepatitis; composite event; multicenter; outcomes; prospective.
\end{abstract}

\section{INTRODUCTION}
The prevalence, healthcare consequences, and economic impact of acute alcohol-associated hepatitis (AH) continue to increase in the United States\supercite{aslam2023,guirguis2015,huang2023,niu2023,sengupta2024}. Currently, AH is the leading cause of liver disease-related hospitalization and emergency room visits and accounts for nearly 20\% of in-hospital mortality\supercite{sengupta2024}. Notably, the definition of AH used in multiple clinical studies also includes patients with underlying cirrhosis\supercite{hosseini2019,osna2017,tu2023,dasarathy2020}. Recent clinical trials have reported conflicting results regarding the impact of pharmacotherapy in severe AH\supercite{forrest2013,thursz2015,szabo2022,gawrieh2024}. The Steroids Or Pentoxifylline in Alcoholic Hepatitis (STOPAH) trial, conducted in Europe, reported 14\%, 30\%, and 57\% mortality at 28 days, 90 days, and 1 year, respectively, in patients treated with steroids\supercite{thursz2015}. Similarly, the Defeat Alcoholic Steatohepatitis (DASH) trial conducted in the United States found mortality rates of 18\%, 42\%, and 44\% at 28, 90, and 180 days, respectively, in patients with severe AH receiving corticosteroid therapy\supercite{szabo2022}. In contrast, a recent trial conducted by AlcHepNet showed corticosteroid therapy coupled with a stopping rule of 7-day Lille score $<0.45^{[6]}$ resulted in much lower mortality of 3\%, 10\%, and 19\% at 30, 90, and 180 days, respectively\supercite{gawrieh2024}, suggesting that use of the Lille score significantly lowers mortality of AH in corticosteroid-treated patients.

Recent epidemiological studies have reported that younger patients and those from racial and ethnic minority groups now account for an increasing proportion of AH cases\supercite{hosseini2019,lee2024}. Sepsis and multiorgan failure have long been recognized as the leading causes of death in AH\supercite{hosseini2019,szabo2022,gawrieh2024,liangpunsakul2011}. Thus, implementing hospital infection control protocols could significantly reduce the risk of adverse clinical outcomes\supercite{kaur2022,diCola2023}. The absence of placebo controls in recent clinical trials also limits our understanding of the natural progression of AH. Notably, two recently completed multicenter randomized trials in AH did not include a placebo arm\supercite{szabo2022,gawrieh2024}. Observational studies generate data that supplement the findings of clinical trials. The Translational Research and Evolving Alcoholic Hepatitis Treatment (TREAT) study, which included 164 patients with AH and 131 concurrently enrolled heavy-drinking controls without liver injury, reported a 1-year mortality of 25\% in AH\supercite{lourens2017}. However, the modest sample size limited a more comprehensive assessment of clinical outcomes in AH. Composite outcomes/major adverse liver outcomes (MALOs) that include important clinical events such as mortality and morbidity (decompensation, hospitalization, and return to drinking) may help determine healthcare utilization and resource allocation, but have not been evaluated in patients with AH. We aimed to identify critical outcomes and generate a composite outcome measure highly relevant to clinical practice.

We present data from a prospective observational study that includes patients with both moderate and severe AH compared to heavy drinking (HD) and healthy controls (HCs). We aimed to generate a composite outcome measure that includes critical outcomes highly relevant to clinical practice in patients with AH. MALOs that include liver-related deaths, liver transplant, new onset development of ascites and HE, and variceal bleeding, have been suggested as relevant to clinical outcomes in patients with metabolic dysfunction–associated steatotic liver disease (MASLD)\supercite{shang2025}. Clinically relevant events in AH also include liver-related hospital admission for worsening ascites, HE, infection, or gastrointestinal bleeding, an increase in MELD score $\ge 5$, and return to drinking within 6 months, as well as MALO. Here we present a novel composite clinical endpoint (AlcHepNet composite index) that offers a comprehensive, clinically relevant analysis of patients with AH for use in designing future trials.

\section{METHODS}

\subsection{Study design}
This prospective observational study was supported by the National Institute on Alcohol Abuse and Alcoholism (NIAAA) and conducted by the AlcHepNet consortium concurrently with their published clinical trial\supercite{tu2023,gawrieh2024}. The study design and diagnosis criteria have been reported in previous publications\supercite{tu2023,dagPerf2024}. A more detailed description of the cohort is provided in the Supplemental Methods and Results section (link placeholder).

Demographic data (age, sex, race and ethnicity, marital status, highest educational level), vital signs, anthropometry, medical history, and concomitant medications; complications of portal hypertension (ascites, jaundice, varices, HE, HCC); laboratory values with calculated Maddrey Discriminant Function (MDF), MELD, aspartate aminotransferase to platelet ratio index (APRI), and Child–Pugh (CP) scores; and HBV, HCV, or HIV status were collected. MELD scores were used to classify AH severity. We defined moderate AH as MELD $<20$ and severe AH as MELD $\ge 20$ based on our prior criteria.

\subsection{Participants}
Detailed inclusion and exclusion criteria for the three groups have been reported elsewhere\supercite{dagPerf2024} and in Supplemental Methods and Results (link placeholder). The study cohort included patients referred from outside medical centers and were over 21 years of age. All research complied with the Declarations of Helsinki and Istanbul, institutional review approvals, and written informed consent. A single IRB (Western IRB--Copernicus Group, Cary, NC) approved the studies (AlcHepNet-01 IRB number 2018323) at all participating sites.

\subsubsection{AH inclusion criteria}
Adult patients with AH defined by NIAAA consensus criteria\supercite{crabb2016} with total bilirubin $>3\,\mathrm{mg/dL}$. A liver biopsy was done in only a small minority of patients ($n=71$; 10\%), all of whom met criteria for definite AH. In the remaining patients ($n=641$; 90\%), the diagnosis of ``Probable'' AH was based primarily on clinical criteria.

\subsubsection{AH exclusion criteria}
Etiologies of liver disease other than alcohol-associated liver disease (ALD) including hemochromatosis, autoimmune liver disease, Wilson disease, and viral hepatitis. The site investigator determined the significance of other liver diseases. None of the patients participated in other clinical trials or received off-label/experimental ALD treatments (e.g., anakinra or G-CSF). Since there are no FDA-approved treatments for AH, corticosteroid use varied by clinical preference and was documented.

\subsubsection{HD inclusion criteria}
History of chronic heavy alcohol consumption sufficient to cause liver damage, i.e., $>40\,\mathrm{g/d}$ or $>280\,\mathrm{g/wk}$ on average for women and $>60\,\mathrm{g/d}$ or $>420\,\mathrm{g/wk}$ on average for men, but without documented liver disease.

\subsubsection{HD exclusion criteria}
Individuals with past evidence of ALD, defined as bilirubin $>2.0\,\mathrm{mg/dL}$, AST $>1.5\times$ ULN, any hospital admission for liver disease, or presence of esophageal varices or ascites (at any time) were excluded. AH patients were recruited in-hospital, HD patients from alcohol-rehabilitation programs, and HC patients by advertisement.

\subsubsection{Alcohol consumption}
Timeline Follow-Back was administered systematically at baseline and each follow-up visit. The full AUDIT was administered at baseline. PEth was not collected systematically. Abstinence was defined as a binary outcome (Yes/No) based on independent assessment by the clinical team.

\subsection{Outcomes}
Protocol-based follow-up was for 180 days, with earlier time points to show that mortality occurred early. Early censoring markers included loss to follow-up, withdrawal, or liver transplantation (LT). Data were reviewed every 24 weeks to capture mortality. All-cause mortality is reported. The \textbf{AlcHepNet composite index} comprised: (1) death (any cause up to 180 d); (2) LT (up to 180 d); (3) new onset of clinically detectable ascites, HE, or variceal bleeding (up to week 24 visit); (4) liver-related hospital admission (ascites, HE, infection, or GI bleeding up to 180 d); (5) increase in MELD $\ge 5$ (to week 24); and (6) return to drinking (RTD) within 6 months.

We also determined ``recompensation'' as a reduction in MELD score $\le 5$ points from enrollment as a surrogate improvement.

\subsection{Statistical analysis}
Continuous data are presented as mean $\pm$ SD; categorical variables as percentages. Quantitative variables were compared using Student's $t$-test or one-way ANOVA. Details are provided in the Supplemental Methods (http://links.lww.com/HEP/K252).

\section{RESULTS}

\subsection{Baseline characteristics of AH and HD}
The study cohort included 712 subjects with AH, 256 with HD, and 159 who were HC (Figure~\ref{fig:enrollment}). The demographics are shown in Table 1 and Supplemental Table S1. As expected, participants with AH had significantly higher bilirubin (total and direct), serum creatinine, AST, ALT, ALP, total white blood cell count (WBC), mean corpuscular volume (MCV), international normalized ratio (INR), prothrombin time (PT), and lower albumin, total protein, hemoglobin, platelet count, estimated glomerular filtration rate (eGFR), and mean arterial pressure (MAP) compared with HD subjects (p values <0.001) (Table~\ref{tab:baseline} and Supplemental Table S1). Before enrollment, 168 patients were started on steroids, but only 57 continued on steroids during the study (Supplemental Table S2). An additional 23 patients started steroids after enrollment. Of the 80 patients who received steroids during the study period, sufficient data were available to evaluate the Day 7 Lille score in 69 (Supplemental Table S3). Most of our cohort had severe AH (79.0\%), characterized by high MELD (21.0\% MELD <20 vs. 31.6\% MELD 20–25, 19.4\% MELD 26–30, and 28.0\% MELD >30), CP (10.3 ± 1.8), and MDF scores (55.1 ± 35.8) (Table 1 and Supplemental Table S1 and Supplemental Figure S1). Although mean APRI scores (3.3 ± 3.4) may suggest that most participants with AH had advanced fibrosis or cirrhosis (Supplemental Figure S1D), the elevated value may be influenced by AST and thrombocytopenia in AH.

\begin{figure}
  \centering
  \includegraphics[width=0.9\textwidth]{figures/fig1.png}
  \caption{Participant enrollment in the AlcHepNet observational study.}
  \label{fig:enrollment}
\end{figure}


% Requires in preamble:
% \usepackage{booktabs,longtable,threeparttablex}

\begin{ThreePartTable}
\begin{TableNotes}
\small
\item Data are presented as mean $\pm$ SD, median (IQR), or $n$ (\%). BMI = body mass index; MELD = Model for End-Stage Liver Disease.
\end{TableNotes}
\begin{longtable}{>{\raggedright\arraybackslash}p{0.4\linewidth} >{\centering\arraybackslash}p{0.2\linewidth} >{\centering\arraybackslash}p{0.2\linewidth} >{\centering\arraybackslash}p{0.2\linewidth}}
\caption{Baseline demographic and clinical characteristics of subjects}\label{tab:baseline}\\
\toprule
\textbf{Characteristic} & \textbf{Subjects with AH} & \textbf{Heavy drinkers without liver disease} & \textbf{Healthy control} \\
 & \textbf{N=712} & \textbf{N=256} & \textbf{N=159} \\
\midrule
\endfirsthead
\caption*{Table~\thetable\ (continued)}\\
\toprule
\textbf{Characteristic} & \textbf{Subjects with AH} & \textbf{Heavy drinkers without liver disease} & \textbf{Healthy control} \\
 & \textbf{N=712} & \textbf{N=256} & \textbf{N=159} \\
\midrule
\endhead
\midrule
\multicolumn{4}{r}{\emph{Continued on next page}}\\
\endfoot
\bottomrule
\insertTableNotes
\endlastfoot
\textbf{Age at OBS enrollment (years)} & 45.0 $\pm$ 10.7 & 48.8 $\pm$ 13.6 & 36.7 $\pm$ 13.5 \\[2pt]
\textbf{Sex} &  &  &  \\
\hspace{1em}Female & 291 (40.9\%) & 109 (42.6\%) & 99 (62.3\%) \\
\hspace{1em}Male & 421 (59.1\%) & 147 (57.4\%) & 60 (37.7\%) \\[2pt]
\textbf{Race} &  &  &  \\
\hspace{1em}Non-White & 95 (13.7\%) & 72 (28.3\%) & 49 (32.7\%) \\
\hspace{1em}White & 598 (86.3\%) & 182 (71.7\%) & 101 (67.3\%) \\[2pt]
\textbf{BMI (kg/m$^2$)} & 29.6 $\pm$ 7.6 & 28.4 $\pm$ 7.5 & 26.4 $\pm$ 5.6 \\
\textbf{BMI (median, IQR)} & 28.3 (24.4, 33.4) & 26.5 (23.4, 31.2) & 25.3 (22.7, 29.8) \\[2pt]
\textbf{BMI category} &  &  &  \\
\hspace{1em}Normal $< 25$ & 183 (28.1\%) & 84 (36.4\%) & 68 (47.2\%) \\
\hspace{1em}Overweight 25--30 & 200 (30.7\%) & 81 (35.1\%) & 44 (30.4\%) \\
\hspace{1em}Obese $\ge 30$ & 291 (41.3\%) & 66 (28.6\%) & 32 (22.4\%) \\[2pt]
Waist circumference (umbilicus, cm) & 107.8 $\pm$ 16.7 & 99.0 $\pm$ 16.6 & 89.0 $\pm$ 15.4 \\
Waist circumference (largest diam., cm) & 109.4 $\pm$ 17.6 & 101.7 $\pm$ 15.9 & 92.1 $\pm$ 15.0 \\
Hip circumference (cm) & 105.0 $\pm$ 16.8 & 105.9 $\pm$ 14.7 & 100.3 $\pm$ 12.8 \\
Mid-upper arm circumference (cm) & 27.4 $\pm$ 5.7 & 30.1 $\pm$ 5.8 & 29.2 $\pm$ 5.0 \\
Waist--hip ratio & 1.0 $\pm$ 0.1 & 0.9 $\pm$ 0.1 & 0.9 $\pm$ 0.1 \\[2pt]
\textbf{Education status} &  &  &  \\
\hspace{1em}No college education & 272 (41.3\%) & 89 (35.3\%) & 7 (4.4\%) \\
\hspace{1em}College education & 387 (58.7\%) & 163 (64.7\%) & 151 (95.6\%) \\[2pt]
\textbf{Marital status} &  &  &  \\
\hspace{1em}Not married & 415 (60.1\%) & 173 (68.1\%) & 104 (65.8\%) \\
\hspace{1em}Married & 275 (39.9\%) & 81 (31.9\%) & 54 (34.2\%) \\[2pt]
Hospitalized for AH within 1 y prior to baseline visit &  &  &  \\
\hspace{1em}No & 441 (63.2\%) & 255 (100.0\%) & 157 (100.0\%) \\
\hspace{1em}Yes & 257 (36.8\%) & 0 (0.0\%) & 0 (0.0\%) \\[2pt]
Alcohol use at baseline &  &  &  \\
\hspace{1em}No & 127 (20.2\%) & 8 (3.2\%) & 81 (53.3\%) \\
\hspace{1em}Yes & 502 (79.8\%) & 244 (96.8\%) & 71 (46.7\%) \\[2pt]
Age at first drink (years) & 18.7 $\pm$ 6.0 & 17.2 $\pm$ 7.2 & 18.9 $\pm$ 3.9 \\
Total number of drinks in past 30 days & 153.1 $\pm$ 197.6 & 238.5 $\pm$ 221.6 & 1.6 $\pm$ 2.5 \\
Total number of drinking days (past 30 days) & 14.9 $\pm$ 11.1 & 22.5 $\pm$ 9.1 & 1.9 $\pm$ 4.8 \\[2pt]
On steroids before or at baseline &  &  &  \\
\hspace{1em}No & 544 (76.4\%) & 252 (98.4\%) & 159 (100.0\%) \\
\hspace{1em}Yes & 168 (23.6\%) & 4 (1.6\%) & 0 (0.0\%) \\[2pt]
On steroids during the study &  &  &  \\
\hspace{1em}No & 632 (88.8\%) & 253 (98.8\%) & 159 (100.0\%) \\
\hspace{1em}Yes & 80 (11.2\%) & 3 (1.2\%) & 0 (0.0\%) \\[2pt]
\textbf{MELD score classification} &  &  &  \\
\hspace{1em}$< 20$ & 148 (21.0\%) & -- & -- \\
\hspace{1em}20--25 & 223 (31.3\%) & -- & -- \\
\hspace{1em}26--30 & 137 (19.4\%) & -- & -- \\
\hspace{1em}$> 30$ & 198 (28.0\%) & -- & -- \\
\end{longtable}
\end{ThreePartTable}



\subsection{Mortality and liver disease severity in AH}
The median length of the follow-up was 154 days (IQR: 34.5–232.5 d). Two hundred thirty-two (33\%) of the 712 patients with AH died during follow-up. Among these, 97 (13.6\%) died within 30 days, 153 (21.5\%) died within 90 days, and 183 (25.7\%) died within 180 days. The total number of deaths in participants with AH was 232, which included those who died after 180 days and those who died after loss to follow-up. We used Kaplan–Meier curves to depict the mortality-time distributions of patient groups stratified by MELD score. Mortality risk was higher with higher MELD scores at both 30 and 90 days (Figure~\ref{fig:km}). The unstratified overall survival and transplant-free survival of patients with AH at 90 days were 76.4\% and 67.0\%, respectively (Supplemental Figures S2A, B).

\begin{figure}
  \centering
  \includegraphics[width=0.9\textwidth]{figures/fig2.png}
  \caption{Kaplan--Meier curves of composite endpoints with and without return to drinking within 180 days among patients with AH.}
  \label{fig:km}
\end{figure}

We further compared the characteristics of those patients with AH who died and those who did not die within 90 days (Supplemental Table S4). Patients with AH who died were more likely to have a higher BMI (31.3±7.7 vs. 29.1±7.5; p=0.002), larger waist circumference (115.6±20.6 cm vs. 108.4±16.9 cm; p=0.012), and elevated MELD score (32±8.0 vs. 25±7.8; p<0.001). They also had higher CP scores (11.0±1.7 vs. 10.1±1.8; p<0.001), MDF scores (76±39.1 vs. 49±32.5; p<0.001), total bilirubin (24.6±12.1 mg/dL vs. 15.5±10.9 mg/dL; p<0.001), serum creatinine (2.2±2.1 mg/dL vs. 1.3±1.5 mg/dL; p<0.001), total WBC count (15.5±10.4 × 109 /L vs. 12.0±7.5 × 109 /L; p<0.001), INR (2.3±0.7 vs. 1.9±0.6; p<0.001), and PT (24.7±7.7 s vs. 20.7±6.1 s; p<0.001). Compared with the patients with AH who survived, those who died had lower eGFR (52.1±32.2 vs. 80.1±39.6; p<0.001), total protein (5.6±0.9 vs. 5.9±0.9; p=0.001), hemoglobin (9.1±1.9 vs. 9.4±1.7; p=0.022), platelet count (117.1±77.8 × 109 /L vs. 142.4±84.5 × 109 /L; p=0.001), MCV (99.4±8.3 vs. 101±8.9 fL; p=0.042), and MAP (78.4±11.6 vs. 84.2±11.2 mmHg; p<0.001).

Among the 232 patients who died, the most frequently observed cause of death in AH was liver decompensation/multiorgan failure, accounting for 47.4\% of the deaths; the next most frequent specified cause of death was infection, including septic shock (7.8\%), followed by renal failure (6.9\%) (Table~\ref{tab:baseline1}). Causes of death were unspecified in 73 of the 232 patients who died (31.5\%). HCC was the least frequent (0.43\%) cause of death in the study cohort.

\begin{ThreePartTable}
\begin{TableNotes}
\small
\item \emph{Note:} Full table available in Supplemental Table S1 (link placeholder). Values are mean $\pm$ SD or $n$ (\%).
\end{TableNotes}
\begin{longtable}{>{\raggedright\arraybackslash}p{0.4\linewidth} >{\centering\arraybackslash}p{0.15\linewidth} >{\centering\arraybackslash}p{0.15\linewidth} >{\centering\arraybackslash}p{0.15\linewidth}}
\caption{Baseline demographic and clinical characteristics of subjects (excerpt).}\label{tab:baseline1}\\
\toprule
\textbf{Characteristic} & \textbf{AH} & \textbf{HD} & \textbf{HC} \\
 & \textbf{N=712} & \textbf{N=256} & \textbf{N=159} \\
\midrule
\endfirsthead
\caption*{Table~\thetable\ (continued)}\\
\toprule
\textbf{Characteristic} & \textbf{AH} & \textbf{HD} & \textbf{HC} \\
 & \textbf{N=712} & \textbf{N=256} & \textbf{N=159} \\
\midrule
\endhead
\midrule
\multicolumn{4}{r}{\emph{Continued on next page}}\\
\endfoot
\bottomrule
\insertTableNotes
\endlastfoot
Age at OBS enrollment (years) & $45.0 \pm 10.7$ & $48.8 \pm 13.2$ & $36.7 \pm 13.5$ \\[2pt]
\textbf{Sex} &  &  &  \\
\hspace{1em}Female & 291 (40.9\%) & 109 (42.6\%) & 99 (62.3\%) \\
\hspace{1em}Male & 421 (59.1\%) & 147 (57.4\%) & 60 (37.7\%) \\[2pt]
White, $n$ (\%) & 598 (86.3\%) & 182 (71.7\%) & 101 (67.3\%) \\[2pt]
BMI (kg/m$^2$) & $29.6 \pm 7.6$ & $28.4 \pm 7.5$ & $26.4 \pm 5.6$ \\[2pt]
Albumin (g/dL) & $2.8 \pm 0.6$ & $4.1 \pm 0.6$ & $4.4 \pm 0.4$ \\
Total bilirubin (mg/dL) & $17.5 \pm 11.8$ & $0.6 \pm 0.3$ & $0.6 \pm 0.4$ \\
Creatinine (mg/dL) & $1.5 \pm 1.7$ & $0.9 \pm 0.3$ & $0.8 \pm 0.2$ \\
AST (IU/L) & $126.9 \pm 73.4$ & $35.2 \pm 61.6$ & $20.4 \pm 5.8$ \\
INR & $2.0 \pm 0.6$ & $1.0 \pm 0.1$ & $1.0 \pm 0.1$ \\
MELD & $26.5 \pm 8.4$ & -- & -- \\
\end{longtable}
\end{ThreePartTable}

The estimated effects of patient characteristics on 90-day mortality, expressed as adjusted hazard ratios (aHRs) from multivariable Cox regression analysis, showed that college or higher levels of education (aHR = 0.551, 95\% CI: [0.352, 0.861], p= 0.0094), ALP (aHR =0.997, 95\% CI:[0.994, 1.000]; p= 0.0328), and MAP (aHR = 0.975, 95\% CI: [0.952, 0.998]; p= 0.0347) were associated with reduced risk of mortality within 90 days. In contrast, higher MELD score (aHR = 1.113, 95\% CI: [1.08, 1.15], p< 0.0001) and total WBC count (aHR =1.026, 95\% CI: [1.003, 1.050]; p =0.0295) were associated with increased mortality risk (Table 3). Interestingly, while survival was not different based on steroid use before or at enrollment, those who received steroids during follow-up had significantly better survival (p < 0.0001 per log-rank test) (Supplemental Figure S3). MELD scores of patients who survived improved over time across all baseline MELD strata (<20, 20–25, 26–30, and $\ge$ 30), with the greatest improvement occurring between weeks 0–4 and 4–12, in those with MELD scores > 30 (Supplemental Figure S4). In contrast, among those patients with AH who died, MELD either remained stable or worsened in 3 of the 4 MELD strata (Supplemental Figure S4). None of the HD patients developed AH or MALO at follow-up.

\subsection{The 3 deaths in HD patients were not due to liver-related causes}
Patients with AH who had decompensated liver disease (as defined in the Methods section) at baseline, compared with those who had compensated liver disease at baseline, were more likely to be White (87.9\% vs. 81.3\%, p = 0.028) and more likely to have been hospitalized due to AH within 1 year before the baseline visit (39.1\% vs. 29.6\%). Compensated patients were more likely to report alcohol use at baseline than decompensated patients (92\% vs. 76\%, p < 0.001) (Supplemental Table S5). As expected, liver disease severity scores, as measured by MELD and CP, and laboratory chemistry and hematologic values were higher in patients with decompensated disease than in those with compensated disease at baseline (Supplemental Table S5).

\section{DISCUSSION}

In this large, prospective, multicenter cohort of contemporary patients with alcohol-associated hepatitis (AH), short-term mortality and morbidity remained substantial. By 180 days, one in four patients had died and nearly two thirds experienced at least one event captured by the AlcHepNet composite index, with events clustering early after enrollment. Mortality scaled with baseline severity: risk was greatest among those with MELD $\ge 20$ and particularly MELD $>30$, while even patients with moderate AH (MELD $<20$) had clinically meaningful 6-month mortality exceeding 8\%. These observations align with prior trials and cohorts that report high early mortality in severe AH and support the prognostic value of MELD in this population\supercite{thursz2015,szabo2022,gawrieh2024,kamath2001,singal2013,dunn2005}.

Multivariable analyses identified higher MELD and leukocytosis as independent predictors of 90-day mortality, whereas higher mean arterial pressure and college education were associated with lower risk. These findings are directionally consistent with prior literature implicating organ dysfunction, systemic inflammation, and hemodynamic instability as key drivers of early death in AH\supercite{hosseini2019,lourens2017,pang2015,forrest2005,kasztelan2013}. The education signal may reflect social determinants of health (access, literacy, resources) rather than biology; nonetheless, it highlights a targetable equity gap within AH care pathways.

Death alone incompletely reflects disease burden and care needs in AH. The AlcHepNet composite index integrates liver decompensation events (ascites, encephalopathy, variceal bleeding), liver-related hospitalization, clinically meaningful laboratory deterioration (MELD increase $\ge 5$), liver transplantation (LT), and early return to drinking (RTD). Including LT recognizes that avoidance of death may still entail substantial morbidity and resource utilization. Incorporating RTD---though not a liver-specific event---captures a proximal behavior strongly linked to subsequent liver outcomes beyond our 180-day window\supercite{louvet2017}. Similar composites are common in other disciplines to enhance power and address competing risks\supercite{pitt2021,bakris2020,sprintrl2015}. In AH, our data show that composite events accrue early (median 62 days including RTD; 92 days excluding RTD), making this endpoint both clinically salient and trial-feasible.

First, the early concentration of events underscores the need for front-loaded, standardized inpatient and immediate post-discharge bundles: infection prevention and early infection recognition\supercite{kaur2022,diCola2023}, hemodynamic optimization, acute kidney injury avoidance, and structured transitions of care. Second, systematic AUD treatment should be embedded alongside hepatology care beginning at index hospitalization, given the frequency and relevance of early RTD\supercite{lee2024,louvet2017}. Third, risk stratification using MELD and inflammatory markers can guide intensity and timing of interventions (e.g., early specialty follow-up, closer laboratory surveillance). Although steroid exposure during follow-up associated with improved survival in our cohort, this likely reflects treatment selection and response (e.g., continuation after a favorable Lille score) rather than a causal effect and should be interpreted cautiously in light of neutral or mixed trial findings\supercite{thursz2015,szabo2022,gawrieh2024}.

LT occurred in 12\% overall (11.5\% by 180 days) with expected enrichment among patients with higher MELD. We also observed early ``recompensation,'' operationalized as MELD improvement $\le 5$ points, in roughly one third of patients. Together, these patterns suggest two divergent short-term trajectories after presentation: rapid deterioration prompting LT versus stabilization/improvement with supportive care. As others have noted, early LT decisions intersect with disease trajectory, abstinence assessment, and inequities in access\supercite{hsu2023,musto2022}. Future prospective work should adjudicate recompensation using standardized clinical criteria (resolution of ascites/HE/bleeding) and evaluate how trajectories map to subsequent transplant candidacy and outcomes\supercite{hofer2023,jophlin2024}.

Strengths include the large sample size, multicenter design, prospective ascertainment across both moderate and severe AH, and the introduction of a pragmatic composite that captures outcomes meaningful to patients and health systems. Limitations include visit-anchored follow-up that precluded precise event timing; potential misclassification of causes of death; nonrandomized, clinician-directed use of steroids and other therapies; binary classification of abstinence without systematic biomarker confirmation (e.g., PEth); and use of MELD change as a surrogate for recompensation. These constraints may bias effect estimates toward the null or introduce residual confounding, and they motivate the need for adjudicated endpoints and more frequent early follow-up in future studies.

The AlcHepNet composite index provides a tractable endpoint for phase II/III AH trials powered for clinically relevant morbidity, not only mortality. Next steps include (1) external validation across diverse health systems; (2) sensitivity analyses weighting component severity and patient relevance; (3) integration with dynamic risk models (e.g., baseline MELD, early change in bilirubin/INR/creatinine, WBC, MAP); (4) formal competing-risk and multi-state modeling to distinguish transitions (stability $\rightarrow$ decompensation $\rightarrow$ LT/death)\supercite{shang2025}; and (5) incorporation of guideline-concordant AUD interventions and patient-reported outcomes\supercite{lee2024,jophlin2024}. Such designs better reflect real-world goals: prevent decompensation, sustain abstinence, reduce admissions, and improve survival and quality of life.

In a modern, U.S. multicenter cohort, AH remains characterized by high early event rates and mortality tightly linked to baseline severity and systemic inflammation. A composite endpoint that includes decompensation, hospitalization, MELD worsening, LT, and early RTD captures this burden and is well suited for evaluating bundled hepatology--addiction care strategies. Preventing early decompensation and RTD should be primary therapeutic targets in forthcoming interventional studies.

\printbibliography

% ---------- End ----------
\end{document}